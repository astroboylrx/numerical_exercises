\section{Introduction}

Partial differential equations (PDEs) are usually grouped into one of
three different classes: {\em hyperbolic}, {\em parabolic}, or {\em
  elliptic}.  You can find the precise mathematical definition of
these classifications in most books on PDEs, but this formal
definition is not very intuitive or useful.  Instead, it is helpful to
look at some prototypical examples of each type of PDE.

When we are solving multiphysics problems, we will see that our 
system of PDEs spans these different types.  Nevertheless, we will
look at solutions methods for each type separately first, and then
use what we learn to solve more complex systems of equations.

\section{Hyperbolic PDEs}

The canonical hyperbolic PDE is the wave equation:
\begin{equation}
  \frac{\partial^2 \phi}{\partial t^2} = c^2 \frac{\partial^2 \phi}{\partial x^2}
\end{equation}   
The general solution to this is traveling waves in either direction:
\begin{equation}
\label{eq:pde:wavesol}
  \phi(x,t) = \alpha f_0(x - ct) + \beta g_0(x + ct)
\end{equation}
Here $f_0$ and $g_0$ are set by the initial
conditions, and the solution propagates $f_0$ to the right and $g_0$ to
the left at a speed $c$.

\begin{exercise}[Wave equation]
{Show by substitution that Eq.~\ref{eq:pde:wavesol} is a solution
to the wave equation}
\end{exercise}

A simple first-order hyperbolic PDE is the linear advection equation:
\begin{equation}
a_t + u a_x = 0
\end{equation}
This simply propagates any initial profile to the right at the speed
$u$.  We will use linear advection as our model equation for numerical
methods for hyperbolic PDEs.

A system of first-order hyperbolic PDEs takes the form:
\begin{equation}
{\bf a}_t + \Ab {\bf a}_x = 0
\end{equation}
where ${\bf a} = (a_0, a_1, \ldots a_{N-1})^\intercal$ and $\Ab$ is a matrix.
This system is hyperbolic if the eigenvalues of $A$ are real (see
\cite{leveque:2002} for an excellent introduction).

An important concept for hyperbolic PDEs are {\em
  characteristics}---these are curves in a space-time diagram along
which the solution is constant.  Associated with these curves is a
speed---this is the wave speed at which information on how the
solution changes is communicated.  For a linear PDE (or system of
PDEs), these will tell you everything you need to know about the
solution.


\section{Elliptic PDEs}

The canonical elliptic PDE is the Poisson equation:
\begin{equation}
  \nabla^2 \phi = f
\end{equation}
Note that there is no time-variable here.  This is a pure boundary
value problem.  The solution, $\phi$ is determined completely by the
source, $f$, and the boundary conditions.

In contrast to the hyperbolic case, there is no propagation of
information here.  The potential, $\phi$, is known instantaneously
everywhere in the domain.  For astrophysical flows, this commonly
arises as the Poisson equation describing the gravitational potential.



\section{Parabolic PDEs}

The canonical parabolic PDE is the heat equation:
\begin{equation}
\label{eq:pde:heat}
  \frac{\partial \phi}{\partial t} = k \frac{\partial^2 f}{\partial x^2}
\end{equation}
This has aspects of both hyperbolic and elliptic PDEs.

The heat equation represents diffusion---an initially sharp feature
will spread out into a smoother profile on a timescale that depends on
the coefficient $k$.  We'll encounter parabolic equations for thermal
diffusion and other types of diffusion (like species, mass), and with
viscosity.


\begin{exercise}[Diffusion timescale]
{Using dimensional analysis, estimate the characteristic timescale for
  diffusion from Eq.~\ref{eq:pde:heat}.}
\end{exercise}
