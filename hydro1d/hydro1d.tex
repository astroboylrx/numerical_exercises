\label{app:hydro1d}

%% \begin{center}
%% \includegraphics[width=0.125\linewidth]{pyro-sm}
%% \end{center}

\section{Introduction}

\hydrooned\ is a one-dimensional compressible hydrodynamics code written
in modern Fortran.  In particular, it implements the piecewise parabolic
method described in \S~\ref{sec:hydro:ppm}, in both Cartesian and spherical
geometries.  It assumes a gamma-law equation of state and supports gravity.


\section{Getting \hydrooned}

\hydrooned\ can be downloaded from its github repository, \url{https://github.com/zingale/hydro1d} as:
\begin{verbatim}
git clone https://github.com/zingale/hydro1d
\end{verbatim}

Some details on the code can be found on its webpage:
\url{http://zingale.github.io/hydro1d/}\, .  As with the other codes detailed here,
if you wish to contribute, fix bugs, etc., you can create issues or pull requests
through the code's github page.



\section{\hydrooned 's structure}

There a few major data structures in \hydrooned, defined in the {\tt
  grid.f90} file:

\begin{itemize}
\item {\tt grid\_t} : this holds the grid information, with fields for
  the low and high indicies of the valid domain, and coordinate
  information (including interface areas and volumes, to support
  spherical geometry).

\item {\tt gridvar\_t} : data that lives on a grid.  This holds both
  a {\tt data} array and the {\tt grid\_t} that it is defined on.
  
\item {\tt gridedgevar\_t} : like {\tt gridvar\_t}, but for data
  defined at the interfaces between zones.

\end{itemize}

The integer keys defined in the {\tt variables\_module} allow you
to index the difference state fields in the data arrays of the 
grid variables.


\section{Running \hydrooned}

Each problem is in it's own directory, and the code is built and run
there.  For example, to run the Sod shock tube problem, do:
\begin{verbatim}
cd hydro1d/sod
make
./hydro1d inputs-sod-xp
\end{verbatim}

As the code is built, object files and modules will be output into the
{\tt \_build} subdirectory.  {\tt make realclean} will clean up the
objects.  Things are setup for {\tt gfortran} by default---you will
need to edit the {\tt Ghydro.mak} with different options for different
compilers. Some bits of Fortran 2003 and 2008 are used, so an
up-to-date compiler is needed.

A number of runtime options can be set---these are listed in {\tt
  params.f90} and {\tt probparams.f90} (the latter is problem-specific
parameters).


\section{Problem setups}

The following problem setups are provided:
\begin{itemize}
\item {\tt advect} : a simple advection test where a density profile
  is advected through periodic boundaries, in a uniform pressure
  medium (to suppress dynamics)

\item {\tt hse} : a simple 1-d atmosphere in hydrostatic equilibrium.
  This test works to see if the atmosphere stays in HSE.

\item {\tt sedov} : a 1-d spherical Sedov explosion.

\item {\tt sod} : a shock tube setup for testing against exact Riemann
  solvers.
\end{itemize}


