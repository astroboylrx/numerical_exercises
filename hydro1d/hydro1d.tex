\label{app:hydro1d}

%% \begin{center}
%% \includegraphics[width=0.125\linewidth]{pyro-sm}
%% \end{center}

\section{Introduction}

\hydrooned\ is a one-dimensional compressible hydrodynamics code written
in modern Fortran.  In particular, it implements the piecewise parabolic
method described in \S~\ref{sec:hydro:ppm}, in both Cartesian and spherical
geometries.  It assumes a gamma-law equation of state.


\section{Getting \hydrooned}

\hydrooned\ can be downloaded from its github repository, \url{https://github.com/zingale/hydro1d} as:
\begin{verbatim}
git clone https://github.com/zingale/hydro1d
\end{verbatim}

Some details on the code can be found on its webpage:
\url{http://zingale.github.io/hydro1d/}\, .



%\section{\hydrooned 's structure}



\section{Running \hydrooned}

Each problem is in it's own directory, and the code is built and run
there.  For example, to run the Sod shock tube problem, do:
\begin{verbatim}
cd hydro1d/sod
make
./hydro1d inputs-sod-xp
\end{verbatim}

As the code is built, object files and modules will be output into the
{\tt \_build} subdirectory.  {\tt make realclean} will clean up the
objects.  Things are setup for {\tt gfortran} by default---you will
need to edit the {\tt Ghydro.mak} with different options for different
compilers. Some bits of Fortran 2003 and 2008 are used, so an
up-to-date compiler is needed.

A number of runtime options can be set---these are listed in {\tt
  params.f90} and {\tt probparams.f90} (the latter is problem-specific
parameters).



